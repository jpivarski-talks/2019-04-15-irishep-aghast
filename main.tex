\pdfminorversion=4
\documentclass[aspectratio=169]{beamer}

\mode<presentation>
{
  \usetheme{default}
  \usecolortheme{default}
  \usefonttheme{default}
  \setbeamertemplate{navigation symbols}{}
  \setbeamertemplate{caption}[numbered]
  \setbeamertemplate{footline}[frame number]  % or "page number"
  \setbeamercolor{frametitle}{fg=white}
  \setbeamercolor{footline}{fg=black}
} 

\usepackage[english]{babel}
\usepackage[utf8x]{inputenc}
\usepackage{tikz}
\usepackage{courier}
\usepackage{array}
\usepackage{bold-extra}
\usepackage{minted}
\usepackage[thicklines]{cancel}
\usepackage{fancyvrb}

\xdefinecolor{dianablue}{rgb}{0.18,0.24,0.31}
\xdefinecolor{darkblue}{rgb}{0.1,0.1,0.7}
\xdefinecolor{darkgreen}{rgb}{0,0.5,0}
\xdefinecolor{darkgrey}{rgb}{0.35,0.35,0.35}
\xdefinecolor{darkorange}{rgb}{0.8,0.5,0}
\xdefinecolor{darkred}{rgb}{0.7,0,0}
\definecolor{darkgreen}{rgb}{0,0.6,0}
\definecolor{mauve}{rgb}{0.58,0,0.82}

\title[2019-04-15-irishep-aghast]{Aghast: communication among histogram implementations}
\author{Jim Pivarski}
\institute{Princeton University -- IRIS-HEP}
\date{April 15, 2019}

\usetikzlibrary{shapes.callouts}

\begin{document}

\logo{\pgfputat{\pgfxy(0.11, 7.4)}{\pgfbox[right,base]{\tikz{\filldraw[fill=dianablue, draw=none] (0 cm, 0 cm) rectangle (50 cm, 1 cm);}\mbox{\hspace{-8 cm}\includegraphics[height=1 cm]{princeton-logo-long.png}\hspace{0.1 cm}\raisebox{0.1 cm}{\includegraphics[height=0.8 cm]{iris-hep-logo-long.png}}\hspace{0.1 cm}}}}}

\begin{frame}
  \titlepage
\end{frame}

\logo{\pgfputat{\pgfxy(0.11, 7.4)}{\pgfbox[right,base]{\tikz{\filldraw[fill=dianablue, draw=none] (0 cm, 0 cm) rectangle (50 cm, 1 cm);}\mbox{\hspace{-8 cm}\includegraphics[height=1 cm]{princeton-logo.png}\hspace{0.1 cm}\raisebox{0.1 cm}{\includegraphics[height=0.8 cm]{iris-hep-logo.png}}\hspace{0.1 cm}}}}}

% Uncomment these lines for an automatically generated outline.
%\begin{frame}{Outline}
%  \tableofcontents
%\end{frame}

% START START START START START START START START START START START START START

\begin{frame}{Context: slide from last summer (8 table rows commented out!)}
\scriptsize
\vspace{0.35 cm}
\begin{columns}
\column{1.1\linewidth}
\renewcommand{\arraystretch}{1.2}
\begin{tabular}{c l c p{2.7 cm} p{1.5 cm} p{4.75 cm}}
pip? & name & last release & interface style & depends on & integrates with \\\hline
& \href{https://root.cern.ch/pyroot}{\textcolor{blue}{PyROOT}} & 2018 & HEP & ROOT & numpy \\
& \href{https://yoda.hepforge.org/pydoc}{\textcolor{blue}{YODA}} & 2018 & HEP & {\it compiled} & matplotlib, yaml \\
$\surd$ & \href{https://pypi.python.org/pypi/physt}{\textcolor{blue}{physt}} & 2018 & HEP + data science & numpy & pandas, xarray, dask, protobuf, matplotlib, vega (plotting), folium (maps) \\
$\surd$ & \href{https://pypi.org/project/fast-histogram}{\textcolor{blue}{fast-histogram}} & 2018 & simple (astronomy) & numpy & \\
$\surd$ & \href{https://pypi.org/project/qhist/}{\textcolor{blue}{qhist}} & 2018 & HEP & ROOT & \\
$\surd$ & \href{https://pypi.org/project/rootpy}{\textcolor{blue}{rootpy}} & 2017 & HEP & ROOT & pytables, matplotlib, stats \\
$\surd$ & \href{https://vaex.io}{\textcolor{blue}{Vaex}} (vaex.io) & 2017 & all-in-one GUI for big data, fast heatmaps & {\it many!} & Jupyter, matplotlib, HDF5, pandas, C++ \\
$\surd$ & \href{https://pypi.python.org/pypi/hdrhistogram}{\textcolor{blue}{hdrhistogram}} & 2017 & ``high dynamic range'' & {\it compiled} & Java, C++ \\
$\surd$ & \href{https://pypi.python.org/pypi/multihist}{\textcolor{blue}{multihist}} & 2017 & numpy wrapper & numpy & matplotlib \\
$\surd$ & \href{https://github.com/ibab/matplotlib-hep}{\textcolor{blue}{matplotlib-hep}} & 2016 & HEP & matplotlib & numpy, scipy \\
$\surd$ & \href{https://pypi.python.org/pypi/pyhistogram}{\textcolor{blue}{pyhistogram}} & 2014 & HEP & numpy & matplotlib, datetime \\
$\surd$ & \href{https://pypi.python.org/pypi/histogram}{\textcolor{blue}{histogram}} & 2011 & HEP & numpy & matplotlib, HDF5 \\
$\surd$ & \href{https://pypi.python.org/pypi/SimpleHist}{\textcolor{blue}{SimpleHist}} & 2011 & HEP & numpy, matplotlib & ROOT \\
$\surd$ & \href{https://pypi.org/project/paida}{\textcolor{blue}{paida}} & 2007 & HEP & & AIDA! \\
& \href{https://github.com/theodoregoetz/histogram}{\textcolor{blue}{theodoregoetz}} & never & HEP & scipy, \mbox{uncertainties} & numpy, matplotlib \\
% $\surd$ & \href{https://pypi.python.org/pypi/histogramy}{\textcolor{blue}{histogramy}} & 2013 & Numpy, Matplotlib, Scikit-Learn & & \\
% $\surd$ & \href{https://pypi.python.org/pypi/pypeaks}{\textcolor{blue}{pypeaks}} & 2014 & & & \\
% $\surd$ & \href{https://pypi.python.org/pypi/hierogram}{\textcolor{blue}{hierogram}} & 2014 & & & \\
% $\surd$ & \href{https://pypi.python.org/pypi/histo}{\textcolor{blue}{histo}} & & & & \\
% $\surd$ & \href{https://pypi.python.org/pypi/python-metrics}{\textcolor{blue}{python-metrics}} & & & & \\
% $\surd$ & \href{https://pypi.python.org/pypi/statscounter}{\textcolor{blue}{statscounter}} & 2016 & & & \\
% $\surd$ & \href{https://pypi.python.org/pypi/datagram}{\textcolor{blue}{datagram}} & & & & \\
% & \href{http://www.ifh.de/~middell/dashi/index.html}{\textcolor{blue}{dashi}} & & & & \\
\end{tabular}
\end{columns}
\end{frame}

\begin{frame}{}
\vspace{0.75 cm}
{\large Over the years, I've written five:}

\scriptsize
\vspace{0.25 cm}
\begin{columns}
\column{1.1\linewidth}
\renewcommand{\arraystretch}{1.2}
\begin{tabular}{c l c p{2.7 cm} p{1.5 cm} p{4.75 cm}}
pip? & name & last release & interface style & depends on & integrates with \\\hline
& \href{http://code.google.com/p/plothon}{\textcolor{blue}{Plothon}} & 2006 & HEP & ROOT & SVG \\
& \href{http://code.google.com/p/svgfig}{\textcolor{blue}{SVGFig}} & 2008 & HEP & {\it nothing!} & SVG \\
& \href{https://github.com/opendatagroup/cassius}{\textcolor{blue}{Cassius}} & 2013 & HEP + data science & numpy & SVG, Augustus (Open Data Group) \\
$\surd$ & \href{https://github.com/histogrammar}{\textcolor{blue}{Histogrammar}} & 2016 & combinational library & numpy & Spark, Julia, CUDA, \mbox{ROOT (Cling),} matplotlib, Bokeh, Vega \\
$\surd$ & \href{https://github.com/scikit-hep/histbook}{\textcolor{blue}{histbook}} & 2018 & HEP & numpy & Spark, Pandas, Vega \\
\end{tabular}
\end{columns}
\normalsize

\vspace{1 cm}
{\large And today, you've heard about three more:}

\scriptsize
\vspace{0.25 cm}
\begin{columns}
\column{1.1\linewidth}
\renewcommand{\arraystretch}{1.2}
\begin{tabular}{c l c p{2.7 cm} p{1.5 cm} p{4.75 cm}}
pip? & name & last release & interface style & depends on & integrates with \\\hline
$\surd$ & \href{https://gitter.im/HSF/mpl-hep}{\textcolor{blue}{mpl-hep}} & 2019 & HEP & Matplotlib & numpy, pandas \\
$\surd$ & \href{https://github.com/boostorg/histogram}{\textcolor{blue}{boost-histogram}} & 2019 & HEP & {\it compiled} & C++, numpy \\
$\surd$ & \href{https://pypi.org/project/hist/}{\textcolor{blue}{hist}} & 2019 & HEP & the above & \\
\end{tabular}
\end{columns}
\end{frame}

\begin{frame}{}
\vspace{1 cm}
\Large
\begin{center}
We're not lacking for options, but perhaps for unity.
\end{center}
\end{frame}

\begin{frame}{}
\large
\vspace{0.75 cm}
Moreover, these libraries are specializing:

\vspace{0.35 cm}
\begin{itemize}
\item {\bf boost-histogram} provides fast and flexible {\it filling},
\item {\bf mpl-hep} provides many {\it plotting} options.
\item Other tools focus on {\it fitting}.
\end{itemize}

\vspace{0.35 cm}
The {\bf hist} package will wrap that up behind a single ``\mintinline{python}{import hist}'' command, but to do so, we need to get histogram objects from one library to another.
\end{frame}

\begin{frame}{Tying it all together}
\large
\vspace{0.5 cm}
The {\bf aghast} library ({\bf ag}gregated, {\bf h}istogram-like {\bf st}atistics) is an {\it in-memory, serializable ontology} of histogram-like objects.

\vspace{0.75 cm}
\includegraphics[width=\linewidth]{aghast-github.png}
\end{frame}

\begin{frame}{}
\vspace{1.25 cm}
\begin{description}\setlength{\itemsep}{0.5 cm}
\item[in-memory:] The purpose is to move histogram-like data from one library to another, invisibly as part of a function call. It is not primarily intended for long-term storage in files. In this respect, it is like Apache Arrow.

\item[serializable:]<2-> We use Flatbuffers as a wire protocol to move data over a network or between languages, if necessary. Flatbuffers supports partial reading/ deserialization. (Flatbuffers is for network games; designed for speed.)

\item[ontology:]<3-> Aghast focuses exclusively on representing the data and converting histograms into histograms. It does not fill or plot them. In this respect, it is like Predictive Model Markup Language (PMML), which neither trains nor scores machine learning models.
\end{description}
\end{frame}

\begin{frame}{Another motivation for the project}
\vspace{0.5 cm}
\large
In IRIS-HEP Analysis Systems, we're developing a \textcolor{darkorange}{\bf Query Service} to keep large datasets inside a system while being responsive to user requests from outside.

\begin{center}
\includegraphics[width=0.6\linewidth]{basic-block-diagram.pdf}
\end{center}

The ``reduced output'' needs to be in some form that can be transmitted, added, and converted into any format. Such an object can be called ``\textcolor{darkorange}{\bf a ghast}.''
\end{frame}

\begin{frame}[fragile]{What does it look like?}
\small
\begin{onlyenv}<1>
\begin{minted}[stripnl=false]{python}
>>> import numpy, aghast
>>> h_numpy = numpy.histogram(numpy.random.normal(0, 1, 100000),
...                           bins=20, range=(-5, 5))

\end{minted}
\end{onlyenv}
\begin{onlyenv}<2->
\begin{minted}{python}
>>> import numpy, aghast
>>> h_numpy = numpy.histogram(numpy.random.normal(0, 1, 100000),
...                           bins=20, range=(-5, 5))
>>> aghast.from_numpy(h_numpy).dump()
\end{minted}
\end{onlyenv}

\scriptsize
\begin{uncoverenv}<3->
\begin{verbatim}
Histogram(
  axis=[
    Axis(binning=RegularBinning(num=20, interval=RealInterval(low=-5.0, high=5.0)))
  ],
  counts=
    UnweightedCounts(
      counts=
        InterpretedInlineInt64Buffer(
          buffer=
              [    2     5    21   107   477  1696  4378  9146 15084 19273 18933 14960
                9216  4425  1648   480   122    23     4     0])))
\end{verbatim}
\end{uncoverenv}

\vspace{0.25 cm}
\uncover<4->{\textcolor{darkblue}{\large Formal class structure, long names $\to$ not for end-users!}}
\end{frame}

\begin{frame}[fragile]{What does it look like?}
\small
\begin{minted}{python}
>>> import ROOT, aghast
>>> h_root = ROOT.TH1F("name", "title", 20, -5, 5)
>>> h_root.FillRandom("gaus", 100000)
>>> aghast.from_root(h_root).dump()
\end{minted}

\uncover<2->{(The ROOT histogram has {\tt moments} to preserve unbinned mean and standard deviation.)}

\tiny
\begin{verbatim}
Histogram(
  axis=[
    Axis(
      binning=
        RegularBinning(
          num=20,
          interval=RealInterval(low=-5.0, high=5.0),
          overflow=RealOverflow(loc_underflow=BinLocation.below1, loc_overflow=BinLocation.above1)),
      statistics=[
        Statistics(
          moments=[
            Moments(sumwxn=InterpretedInlineInt64Buffer(buffer=[100000]), n=0),
            Moments(sumwxn=InterpretedInlineFloat64Buffer(buffer=[100000]), n=0, weightpower=1),
            Moments(sumwxn=InterpretedInlineFloat64Buffer(buffer=[100000]), n=0, weightpower=2),
            Moments(sumwxn=InterpretedInlineFloat64Buffer(buffer=[-102.751]), n=1, weightpower=1),
            Moments(sumwxn=InterpretedInlineFloat64Buffer(buffer=[104061]), n=2, weightpower=1)
          ])
      ])
  ],
  counts=
    UnweightedCounts(
      counts=
        InterpretedInlineBuffer(
          buffer=
              [0.0000e+00 0.0000e+00 4.0000e+00 1.8000e+01 1.0000e+02 4.7900e+02
               1.6950e+03 4.3380e+03 9.2950e+03 1.4974e+04 1.9195e+04 1.9289e+04
               1.4781e+04 9.1450e+03 4.3850e+03 1.6680e+03 4.9200e+02 1.1600e+02
               2.3000e+01 3.0000e+00 0.0000e+00 0.0000e+00],
          dtype=Interpretation.float32)),
  title='title')
\end{verbatim}
\end{frame}

\begin{frame}[fragile]{What does it look like?}
\small
{\large ROOT $\to$ Aghast $\to$ Numpy:}
\begin{minted}{python}
>>> aghast.to_numpy(aghast.from_root(h_root))
\end{minted}

\scriptsize
\begin{verbatim}
(array([0.0000e+00, 0.0000e+00, 4.0000e+00, 1.8000e+01, 1.0000e+02,
        4.7900e+02, 1.6950e+03, 4.3380e+03, 9.2950e+03, 1.4974e+04,
        1.9195e+04, 1.9289e+04, 1.4781e+04, 9.1450e+03, 4.3850e+03,
        1.6680e+03, 4.9200e+02, 1.1600e+02, 2.3000e+01, 3.0000e+00,
        0.0000e+00, 0.0000e+00], dtype=float32),
 array([-inf, -5. , -4.5, -4. , -3.5, -3. , -2.5, -2. , -1.5, -1. , -0.5,
         0. ,  0.5,  1. ,  1.5,  2. ,  2.5,  3. ,  3.5,  4. ,  4.5,  5. ,
         inf]))
\end{verbatim}

\uncover<2->{\large Even though Numpy doesn't have a concept of regularly spaced bins or overflow bins, we can shoehorn the ROOT histogram into this form.}
\end{frame}

\begin{frame}[fragile]{What does it look like?}
\vspace{0.5 cm}
\small
{\large Numpy $\to$ Aghast $\to$ ROOT:}
\begin{minted}{python}
>>> aghast.to_root(aghast.from_numpy(h_numpy), "name").Draw()
\end{minted}

\begin{center}
\includegraphics[width=0.65\linewidth]{c1.png}
\end{center}
\end{frame}

\begin{frame}[fragile]{What does it look like?}
\vspace{0.25 cm}
Add histograms with different structures:

\small
\begin{minted}{python}
>>> (aghast.from_numpy(h_numpy) + aghast.from_root(h_root)).dump()
\end{minted}

\scriptsize
\begin{verbatim}
Histogram(
  axis=[
    Axis(
      binning=
        RegularBinning(
          num=20,
          interval=RealInterval(low=-5.0, high=5.0),
          overflow=RealOverflow(loc_underflow=BinLocation.above1,
                                loc_overflow=BinLocation.above2)))
  ],
  counts=
    UnweightedCounts(
      counts=
        InterpretedInlineFloat64Buffer(
          buffer=
              [0.0000e+00 5.0000e+00 3.6000e+01 2.1700e+02 9.6800e+02 3.2320e+03
               8.7780e+03 1.8562e+04 2.9957e+04 3.8380e+04 3.8551e+04 2.9654e+04
               1.8311e+04 8.7890e+03 3.2790e+03 1.0010e+03 2.4000e+02 3.4000e+01
               6.0000e+00 0.0000e+00 0.0000e+00 0.0000e+00])))
\end{verbatim}
\end{frame}

\begin{frame}[fragile]{What does it look like?}
\vspace{0.25 cm}
\small
In Pandas, binnings become indexes and per-bin statistics (only one here) become columns.
\begin{minted}{python}
>>> aghast.to_pandas(aghast.from_root(h_root))
\end{minted}

\tiny
\begin{verbatim}
              unweighted
[-inf, -5.0)         0.0
[-5.0, -4.5)         0.0
[-4.5, -4.0)         4.0
[-4.0, -3.5)        18.0
[-3.5, -3.0)       100.0
[-3.0, -2.5)       479.0
[-2.5, -2.0)      1695.0
[-2.0, -1.5)      4338.0
[-1.5, -1.0)      9295.0
[-1.0, -0.5)     14974.0
[-0.5, 0.0)      19195.0
[0.0, 0.5)       19289.0
[0.5, 1.0)       14781.0
[1.0, 1.5)        9145.0
[1.5, 2.0)        4385.0
[2.0, 2.5)        1668.0
[2.5, 3.0)         492.0
[3.0, 3.5)         116.0
[3.5, 4.0)          23.0
[4.0, 4.5)           3.0
[4.5, 5.0)           0.0
[5.0, inf)           0.0
\end{verbatim}
\end{frame}

%% >>> tmessage = ROOT.TMessage()
%% >>> tmessage.WriteObject(h_root)
%% >>> tmessage.Length()
%% 642
%% >>> len(aghast.from_root(h_root).tobuffer())
%% 544

\begin{frame}{Common way of describing features in all histogram libraries}
\vspace{0.25 cm}
\scriptsize
\begin{columns}[t]
\column{0.33\linewidth}
\begin{itemize}
  \item \href{https://github.com/scikit-hep/aghast/blob/master/specification.adoc\#collection}{\textcolor{blue}{Collection}}
  \item \href{https://github.com/scikit-hep/aghast/blob/master/specification.adoc\#histogram}{\textcolor{blue}{Histogram}}
  \item \href{https://github.com/scikit-hep/aghast/blob/master/specification.adoc\#axis}{\textcolor{blue}{Axis}}
  \item \href{https://github.com/scikit-hep/aghast/blob/master/specification.adoc\#integerbinning}{\textcolor{blue}{IntegerBinning}}
  \item \href{https://github.com/scikit-hep/aghast/blob/master/specification.adoc\#regularbinning}{\textcolor{blue}{RegularBinning}}
  \item \href{https://github.com/scikit-hep/aghast/blob/master/specification.adoc\#realinterval}{\textcolor{blue}{RealInterval}}
  \item \href{https://github.com/scikit-hep/aghast/blob/master/specification.adoc\#realoverflow}{\textcolor{blue}{RealOverflow}}
  \item \href{https://github.com/scikit-hep/aghast/blob/master/specification.adoc\#hexagonalbinning}{\textcolor{blue}{HexagonalBinning}}
  \item \href{https://github.com/scikit-hep/aghast/blob/master/specification.adoc\#edgesbinning}{\textcolor{blue}{EdgesBinning}}
  \item \href{https://github.com/scikit-hep/aghast/blob/master/specification.adoc\#irregularbinning}{\textcolor{blue}{IrregularBinning}}
  \item \href{https://github.com/scikit-hep/aghast/blob/master/specification.adoc\#categorybinning}{\textcolor{blue}{CategoryBinning}}
  \item \href{https://github.com/scikit-hep/aghast/blob/master/specification.adoc\#sparseregularbinning}{\textcolor{blue}{SparseRegularBinning}}
  \item \href{https://github.com/scikit-hep/aghast/blob/master/specification.adoc\#fractionbinning}{\textcolor{blue}{FractionBinning}}
  \item \href{https://github.com/scikit-hep/aghast/blob/master/specification.adoc\#predicatebinning}{\textcolor{blue}{PredicateBinning}}
  \item \href{https://github.com/scikit-hep/aghast/blob/master/specification.adoc\#variationbinning}{\textcolor{blue}{VariationBinning}}
\end{itemize}
\column{0.33\linewidth}
\begin{itemize}
  \item \href{https://github.com/scikit-hep/aghast/blob/master/specification.adoc\#variation}{\textcolor{blue}{Variation}}
  \item \href{https://github.com/scikit-hep/aghast/blob/master/specification.adoc\#assignment}{\textcolor{blue}{Assignment}}
  \item \href{https://github.com/scikit-hep/aghast/blob/master/specification.adoc\#unweightedcounts}{\textcolor{blue}{UnweightedCounts}}
  \item \href{https://github.com/scikit-hep/aghast/blob/master/specification.adoc\#weightedcounts}{\textcolor{blue}{WeightedCounts}}
  \item \href{https://github.com/scikit-hep/aghast/blob/master/specification.adoc\#interpretedinlinebuffer}{\textcolor{blue}{InterpretedInlineBuffer}}
  \item \href{https://github.com/scikit-hep/aghast/blob/master/specification.adoc\#interpretedinlineint64buffer}{\textcolor{blue}{InterpretedInlineInt64Buffer}}
  \item \href{https://github.com/scikit-hep/aghast/blob/master/specification.adoc\#interpretedinlinefloat64buffer}{\textcolor{blue}{InterpretedInlineFloat64Buffer}}
  \item \href{https://github.com/scikit-hep/aghast/blob/master/specification.adoc\#interpretedexternalbuffer}{\textcolor{blue}{InterpretedExternalBuffer}}
  \item \href{https://github.com/scikit-hep/aghast/blob/master/specification.adoc\#profile}{\textcolor{blue}{Profile}}
  \item \href{https://github.com/scikit-hep/aghast/blob/master/specification.adoc\#statistics}{\textcolor{blue}{Statistics}}
  \item \href{https://github.com/scikit-hep/aghast/blob/master/specification.adoc\#moments}{\textcolor{blue}{Moments}}
  \item \href{https://github.com/scikit-hep/aghast/blob/master/specification.adoc\#quantiles}{\textcolor{blue}{Quantiles}}
  \item \href{https://github.com/scikit-hep/aghast/blob/master/specification.adoc\#modes}{\textcolor{blue}{Modes}}
  \item \href{https://github.com/scikit-hep/aghast/blob/master/specification.adoc\#extremes}{\textcolor{blue}{Extremes}}
  \item \href{https://github.com/scikit-hep/aghast/blob/master/specification.adoc\#statisticfilter}{\textcolor{blue}{StatisticFilter}}
\end{itemize}
\column{0.33\linewidth}
\begin{itemize}
  \item \href{https://github.com/scikit-hep/aghast/blob/master/specification.adoc\#covariance}{\textcolor{blue}{Covariance}}
  \item \href{https://github.com/scikit-hep/aghast/blob/master/specification.adoc\#parameterizedfunction}{\textcolor{blue}{ParameterizedFunction}}
  \item \href{https://github.com/scikit-hep/aghast/blob/master/specification.adoc\#parameter}{\textcolor{blue}{Parameter}}
  \item \href{https://github.com/scikit-hep/aghast/blob/master/specification.adoc\#evaluatedfunction}{\textcolor{blue}{EvaluatedFunction}}
  \item \href{https://github.com/scikit-hep/aghast/blob/master/specification.adoc\#binnedevaluatedfunction}{\textcolor{blue}{BinnedEvaluatedFunction}}
  \item \href{https://github.com/scikit-hep/aghast/blob/master/specification.adoc\#ntuple}{\textcolor{blue}{Ntuple}}
  \item \href{https://github.com/scikit-hep/aghast/blob/master/specification.adoc\#column}{\textcolor{blue}{Column}}
  \item \href{https://github.com/scikit-hep/aghast/blob/master/specification.adoc\#ntupleinstance}{\textcolor{blue}{NtupleInstance}}
  \item \href{https://github.com/scikit-hep/aghast/blob/master/specification.adoc\#chunk}{\textcolor{blue}{Chunk}}
  \item \href{https://github.com/scikit-hep/aghast/blob/master/specification.adoc\#columnchunk}{\textcolor{blue}{ColumnChunk}}
  \item \href{https://github.com/scikit-hep/aghast/blob/master/specification.adoc\#page}{\textcolor{blue}{Page}}
  \item \href{https://github.com/scikit-hep/aghast/blob/master/specification.adoc\#rawinlinebuffer}{\textcolor{blue}{RawInlineBuffer}}
  \item \href{https://github.com/scikit-hep/aghast/blob/master/specification.adoc\#rawexternalbuffer}{\textcolor{blue}{RawExternalBuffer}}
  \item \href{https://github.com/scikit-hep/aghast/blob/master/specification.adoc\#metadata}{\textcolor{blue}{Metadata}}
  \item \href{https://github.com/scikit-hep/aghast/blob/master/specification.adoc\#decoration}{\textcolor{blue}{Decoration}}
\end{itemize}
\end{columns}
\end{frame}

\begin{frame}{Common way of describing features in all histogram libraries}
\vspace{0.5 cm}
\begin{itemize}
\item Only one {\bf Histogram} type; it's $n$-dimensional and slicable like a Numpy array.
\item Regular/irregular/string-labeled/sparse axis types are all {\bf Binnings} (and they become an index or multi-index in Pandas).
\item Efficiency plots are plots that have a {\bf FractionBinning} in some axis.
\item {\bf Collections} of plots representing the same observables with different cuts share a {\bf PredicateBinning}.
\item {\bf Collections} of plots representing different systematic variations share a {\bf VariationBinning}.
\item An {\bf Axis} may also have {\bf Moments} and {\bf Correlations}.
\item {\bf Profile} plots are those that have binned {\bf Moments}. Many {\bf Profiles} can share the same binning (and they become columns in Pandas).
\item {\bf Functions} may be attached to {\bf Histograms} or may be unattached in {\bf Collections}.
\item {\bf Collections} may also contain simple {\bf Ntuples} for unbinned fitting.
\end{itemize}
\end{frame}

\begin{frame}{}
\LARGE
\begin{center}
\textcolor{darkblue}{Extracting bin values and \\ histogram-to-histogram transformations}
\end{center}
\end{frame}

\begin{frame}[fragile]{Numpy-like slicing with overflow bins}
\scriptsize
\vspace{0.4 cm}
\begin{onlyenv}<1>
\begin{minted}{python}
>>> allcounts = numpy.ones((7, 7), int)
>>> allcounts[5, :] = allcounts[6, :] = allcounts[:, 0] = allcounts[:, 1] = 999

>>> h = aghast.Histogram(
...     [aghast.Axis(aghast.RegularBinning(5, aghast.RealInterval(-5, 5),
...                  aghast.RealOverflow(loc_underflow=aghast.RealOverflow.above1,
...                                      loc_overflow=aghast.RealOverflow.above2))),
...      aghast.Axis(aghast.RegularBinning(5, aghast.RealInterval(-5, 5),
...                  aghast.RealOverflow(loc_underflow=aghast.RealOverflow.below2,
...                                      loc_overflow=aghast.RealOverflow.below1)))],
...     aghast.UnweightedCounts(
...         aghast.InterpretedInlineBuffer.fromarray(allcounts)))

>>> h.counts[:, :]                         # normal slice gets the non-overflow bins
\end{minted}
\begin{verbatim}
array([[1, 1, 1, 1, 1],
       [1, 1, 1, 1, 1],
       [1, 1, 1, 1, 1],
       [1, 1, 1, 1, 1],
       [1, 1, 1, 1, 1]])


\end{verbatim}
\end{onlyenv}
\begin{onlyenv}<2>
\begin{minted}{python}
>>> allcounts = numpy.ones((7, 7), int)
>>> allcounts[5, :] = allcounts[6, :] = allcounts[:, 0] = allcounts[:, 1] = 999

>>> h = aghast.Histogram(
...     [aghast.Axis(aghast.RegularBinning(5, aghast.RealInterval(-5, 5),
...                  aghast.RealOverflow(loc_underflow=aghast.RealOverflow.above1,
...                                      loc_overflow=aghast.RealOverflow.above2))),
...      aghast.Axis(aghast.RegularBinning(5, aghast.RealInterval(-5, 5),
...                  aghast.RealOverflow(loc_underflow=aghast.RealOverflow.below2,
...                                      loc_overflow=aghast.RealOverflow.below1)))],
...     aghast.UnweightedCounts(
...         aghast.InterpretedInlineBuffer.fromarray(allcounts)))

>>> h.counts[-numpy.inf:numpy.inf, :]      # infinity at one end means under/overflow
\end{minted}
\begin{verbatim}
array([[999, 999, 999, 999, 999], 
       [  1,   1,   1,   1,   1], 
       [  1,   1,   1,   1,   1], 
       [  1,   1,   1,   1,   1], 
       [  1,   1,   1,   1,   1], 
       [  1,   1,   1,   1,   1], 
       [999, 999, 999, 999, 999]])
\end{verbatim}
\end{onlyenv}
\begin{onlyenv}<3>
\begin{minted}{python}
>>> allcounts = numpy.ones((7, 7), int)
>>> allcounts[5, :] = allcounts[6, :] = allcounts[:, 0] = allcounts[:, 1] = 999

>>> h = aghast.Histogram(
...     [aghast.Axis(aghast.RegularBinning(5, aghast.RealInterval(-5, 5),
...                  aghast.RealOverflow(loc_underflow=aghast.RealOverflow.above1,
...                                      loc_overflow=aghast.RealOverflow.above2))),
...      aghast.Axis(aghast.RegularBinning(5, aghast.RealInterval(-5, 5),
...                  aghast.RealOverflow(loc_underflow=aghast.RealOverflow.below2,
...                                      loc_overflow=aghast.RealOverflow.below1)))],
...     aghast.UnweightedCounts(
...         aghast.InterpretedInlineBuffer.fromarray(allcounts)))

>>> h.counts[-numpy.inf:numpy.inf, -numpy.inf:]
\end{minted}
\begin{verbatim}
array([[999, 999, 999, 999, 999, 999], 
       [999,   1,   1,   1,   1,   1], 
       [999,   1,   1,   1,   1,   1], 
       [999,   1,   1,   1,   1,   1], 
       [999,   1,   1,   1,   1,   1], 
       [999,   1,   1,   1,   1,   1], 
       [999, 999, 999, 999, 999, 999]])
\end{verbatim}
\end{onlyenv}
\end{frame}

\begin{frame}[fragile]{Pandas-like loc/iloc slicing}
\scriptsize
\vspace{0.4 cm}
\begin{onlyenv}<1>
\begin{minted}{python}
>>> allcounts = numpy.ones((7, 7), int)
>>> allcounts[5, :] = allcounts[6, :] = allcounts[:, 0] = allcounts[:, 1] = 999

>>> h = aghast.Histogram(
...     [aghast.Axis(aghast.RegularBinning(5, aghast.RealInterval(-5, 5),
...                  aghast.RealOverflow(loc_underflow=aghast.RealOverflow.above1,
...                                      loc_overflow=aghast.RealOverflow.above2))),
...      aghast.Axis(aghast.RegularBinning(5, aghast.RealInterval(-5, 5),
...                  aghast.RealOverflow(loc_underflow=aghast.RealOverflow.below2,
...                                      loc_overflow=aghast.RealOverflow.below1)))],
...     aghast.UnweightedCounts(
...         aghast.InterpretedInlineBuffer.fromarray(allcounts)))

>>> h.loc[0.0:, -5.0:4.5].counts[:, :]     # loc slices to make a new histogram
\end{minted}
\begin{verbatim}
array([[1, 1, 1, 1, 1], 
       [1, 1, 1, 1, 1], 
       [1, 1, 1, 1, 1]])




\end{verbatim}
\end{onlyenv}
\begin{onlyenv}<2>
\begin{minted}{python}
>>> allcounts = numpy.ones((7, 7), int)
>>> allcounts[5, :] = allcounts[6, :] = allcounts[:, 0] = allcounts[:, 1] = 999

>>> h = aghast.Histogram(
...     [aghast.Axis(aghast.RegularBinning(5, aghast.RealInterval(-5, 5),
...                  aghast.RealOverflow(loc_underflow=aghast.RealOverflow.above1,
...                                      loc_overflow=aghast.RealOverflow.above2))),
...      aghast.Axis(aghast.RegularBinning(5, aghast.RealInterval(-5, 5),
...                  aghast.RealOverflow(loc_underflow=aghast.RealOverflow.below2,
...                                      loc_overflow=aghast.RealOverflow.below1)))],
...     aghast.UnweightedCounts(
...         aghast.InterpretedInlineBuffer.fromarray(allcounts)))

>>> h.iloc[2:, 0:5].counts[:, :]           # iloc slices with index coordinates
\end{minted}
\begin{verbatim}
array([[1, 1, 1, 1, 1], 
       [1, 1, 1, 1, 1], 
       [1, 1, 1, 1, 1]])




\end{verbatim}
\end{onlyenv}
\begin{onlyenv}<3>
\begin{minted}{python}
>>> allcounts = numpy.ones((7, 7), int)
>>> allcounts[5, :] = allcounts[6, :] = allcounts[:, 0] = allcounts[:, 1] = 999

>>> h = aghast.Histogram(
...     [aghast.Axis(aghast.RegularBinning(5, aghast.RealInterval(-5, 5),
...                  aghast.RealOverflow(loc_underflow=aghast.RealOverflow.above1,
...                                      loc_overflow=aghast.RealOverflow.above2))),
...      aghast.Axis(aghast.RegularBinning(5, aghast.RealInterval(-5, 5),
...                  aghast.RealOverflow(loc_underflow=aghast.RealOverflow.below2,
...                                      loc_overflow=aghast.RealOverflow.below1)))],
...     aghast.UnweightedCounts(
...         aghast.InterpretedInlineBuffer.fromarray(allcounts)))

>>> h.iloc[::2, ::2].counts[:, :]          # ::2 rebins by a factor of 2
\end{minted}
\begin{verbatim}
array([[4, 4], 
       [4, 4]])





\end{verbatim}
\end{onlyenv}
\begin{onlyenv}<4>
\begin{minted}{python}
>>> allcounts = numpy.ones((7, 7), int)
>>> allcounts[5, :] = allcounts[6, :] = allcounts[:, 0] = allcounts[:, 1] = 999

>>> h = aghast.Histogram(
...     [aghast.Axis(aghast.RegularBinning(5, aghast.RealInterval(-5, 5),
...                  aghast.RealOverflow(loc_underflow=aghast.RealOverflow.above1,
...                                      loc_overflow=aghast.RealOverflow.above2))),
...      aghast.Axis(aghast.RegularBinning(5, aghast.RealInterval(-5, 5),
...                  aghast.RealOverflow(loc_underflow=aghast.RealOverflow.below2,
...                                      loc_overflow=aghast.RealOverflow.below1)))],
...     aghast.UnweightedCounts(
...         aghast.InterpretedInlineBuffer.fromarray(allcounts)))

>>> h.loc[0.0:, None].counts[:]            # None projects (sums over) a dimension
\end{minted}
\begin{verbatim}
array([2003, 2003, 2003])






\end{verbatim}
\end{onlyenv}
\end{frame}

\begin{frame}{}
\vspace{1 cm}
\large
\begin{center}
{\bf Note:} slicing, projecting, and rebinning are aspects of the same operation:

\vspace{0.5 cm}
\begin{minipage}{0.9\linewidth}
\begin{itemize}
\item {\tt \textcolor{darkorange}{[}slice-from\textcolor{darkorange}{:}slice-to\textcolor{darkorange}{:}rebin-by\textcolor{darkorange}{]}} (like Python/Numpy)
\item {\tt \textcolor{darkorange}{[}..., None, ...\textcolor{darkorange}{]}} projection: sum over and remove dimension
\item {\tt loc\textcolor{darkorange}{[}...\textcolor{darkorange}{]}} in x-axis coordinates
\item {\tt iloc\textcolor{darkorange}{[}...\textcolor{darkorange}{]}} in integer indexes
\end{itemize}
\end{minipage}
\end{center}
\end{frame}

\begin{frame}[fragile]{Numpy advanced indexing: slices that leave gaps}
\scriptsize
\vspace{0.4 cm}
\begin{onlyenv}<1>
\begin{minted}{python}
>>> allcounts = numpy.ones((7, 7), int)
>>> allcounts[5, :] = allcounts[6, :] = allcounts[:, 0] = allcounts[:, 1] = 999

>>> h = aghast.Histogram(
...     [aghast.Axis(aghast.RegularBinning(5, aghast.RealInterval(-5, 5),
...                  aghast.RealOverflow(loc_underflow=aghast.RealOverflow.above1,
...                                      loc_overflow=aghast.RealOverflow.above2))),
...      aghast.Axis(aghast.RegularBinning(5, aghast.RealInterval(-5, 5),
...                  aghast.RealOverflow(loc_underflow=aghast.RealOverflow.below2,
...                                      loc_overflow=aghast.RealOverflow.below1)))],
...     aghast.UnweightedCounts(
...         aghast.InterpretedInlineBuffer.fromarray(allcounts)))

>>> h.iloc[[True, False, True, False, True], None].axis[0].binning.dump()
\end{minted}
\begin{verbatim}
IrregularBinning(
  intervals=[
    RealInterval(low=-5.0, high=-3.0),
    RealInterval(low=-1.0, high=1.0),
    RealInterval(low=3.0, high=5.0)
  ],
  overflow=RealOverflow(loc_underflow=BinLocation.above1, loc_overflow=BinLocation.above2))
\end{verbatim}
\end{onlyenv}






\end{frame}

\begin{frame}{Summary}
\large
\vspace{0.4 cm}
\begin{itemize}\setlength{\itemsep}{0.3 cm}
\item Instead of writing $\frac{N(N - 1)}{2}$ converters among $N$ histogramming libraries, create one \textcolor{darkorange}{\bf ontology} and convert the $N$ libraries into and out of it.

\item Allows Python histogramming libraries to specialize: a user can access all features without one library implementing all features.

\item Provides a ``reduced output'' format for a future \textcolor{darkorange}{\bf Query Service} without tying it to any particular histogramming library.

\item Histograms are $n$-dimensional and \textcolor{darkorange}{\bf sliced/projected/rebinned} like Numpy.

\item Choice of \textcolor{darkorange}{\bf Flatbuffers}:

\vspace{0.25 cm}
\begin{itemize}\setlength{\itemsep}{0.25 cm}
\item {\large Minimal wire protocol, can be selectively deserialized (like {\tt\small TDirectory}).}
\item {\large Schema evolution: we can add classes as needed.}
\item {\large Multilingual: supports 12 languages (most development in C++).}
\end{itemize}
\end{itemize}
\end{frame}

\end{document}
